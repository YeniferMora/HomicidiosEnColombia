\documentclass[lettersize,journal]{IEEEtran}
\usepackage{amsmath,amsfonts}
\usepackage{algorithmic}
\usepackage{algorithm}
\usepackage{array}
\usepackage[caption=false,font=normalsize,labelfont=sf,textfont=sf]{subfig}
\usepackage{textcomp}
\usepackage{stfloats}
\usepackage{url}
\usepackage{verbatim}
\usepackage{graphicx}
\usepackage{cite}
\usepackage{tabularx}

\hyphenation{op-tical net-works semi-conduc-tor IEEE-Xplore}
% updated with editorial comments 8/9/2021

\begin{document}

\title{A Sample Article Using IEEEtran.cls\\ for IEEE Journals and Transactions}

\author{IEEE Publication Technology,~\IEEEmembership{Staff,~IEEE,}
        % <-this % stops a space
\thanks{This paper was produced by the IEEE Publication Technology Group. They are in Piscataway, NJ.}% <-this % stops a space
\thanks{Manuscript received April 19, 2021; revised August 16, 2021.}}

% The paper headers
\markboth{Journal of \LaTeX\ Class Files,~Vol.~14, No.~8, August~2021}%
{Shell \MakeLowercase{\textit{et al.}}: A Sample Article Using IEEEtran.cls for IEEE Journals}

\IEEEpubid{0000--0000/00\$00.00~\copyright~2021 IEEE}
% Remember, if you use this you must call \IEEEpubidadjcol in the second
% column for its text to clear the IEEEpubid mark.

\maketitle

\begin{abstract}
This document describes the most common article elements and how to use the IEEEtran class with \LaTeX \ 
\end{abstract}

\begin{IEEEkeywords}
Article submission, IEEE, IEEEtran, journal, \LaTeX, paper, template, typesetting.
\end{IEEEkeywords}

\section{Introduction}
\IEEEPARstart{T}{his} file is intended to serve as a ``sample article file''
for IEEE journal papers produced under \LaTeX\ using
IEEEtran.cls version 1.8b and later. 

%---------------------------------------------------------------------------------
% Secciones del Documento
%---------------------------------------------------------------------------------

\section{Descripción del Conjunto de Datos}

El conjunto de datos utilizado en este proyecto se titula \textit{Presuntos homicidios. Colombia, 2015 a 2023}, recopilado por el Instituto Nacional de Medicina Legal y Ciencias Forenses (INMLCF). Está enfocado en estadísticas criminales y forenses, relacionadas con homicidios presuntos ocurridos en Colombia.

\subsection{Dimensionalidad}
\begin{itemize}
    \item \textbf{Variables:} 35 columnas, incluyendo variables de identificación, características de la víctima (edad, sexo, escolaridad), contexto temporal (año, mes, día, rango de hora) y geográfico (código DANE de municipio).
    \item \textbf{Registros:} 111263 casos, uno por cada presunto homicidio.
    \item \textbf{Ventana de tiempo:} Desde 2015 hasta 2023.
\end{itemize}

\subsection{Descripción de las Variables}

Descripción detalla en tabla 1

\begin{table*}[htbp]
  \caption{Resumen estadístico de variables del conjunto de datos}
  \centering
  \begin{tabularx}{\textwidth}{|c|X|c|c|c|c|c|}
  \hline
  \textbf{Nº} & \textbf{Columna} & \textbf{Tipo de Dato} & \textbf{Mínimo} & \textbf{Máximo} & \textbf{Rango} & \textbf{Valores Únicos} \\
  \hline
  1 & ID & Numérico & 1 & 111263 & 111262 & 111263 \\
  2 & Año del hecho & Numérico & 2015 & 2023 & 8 & 9 \\
  3 & Grupo de edad de la víctima & Categórico & - & - & - & 19 \\
  4 & Grupo Mayor Menor de Edad & Categórico & - & - & - & 3 \\
  5 & Edad judicial & Categórico & - & - & - & 19 \\
  6 & Ciclo Vital & Categórico & - & - & - & 7 \\
  7 & Sexo de la víctima & Categórico & - & - & - & 3 \\
  8 & Estado Civil & Categórico & - & - & - & 8 \\
  9 & País de Nacimiento de la Víctima & Categórico & - & - & - & 55 \\
  10 & Escolaridad & Categórico & - & - & - & 12 \\
  11 & Pertenencia Grupal & Categórico & - & - & - & 29 \\
  12 & Mes del hecho & Categórico & - & - & - & 13 \\
  13 & Día del hecho & Categórico & - & - & - & 8 \\
  14 & Rango de Hora del Hecho X 3 Horas & Categórico & - & - & - & 10 \\
  15 & Código Dane Municipio & Numérico & 999 & 99773 & 98774 & 1075 \\
  16 & Municipio del hecho DANE & Categórico & - & - & - & 1039 \\
  17 & Departamento del hecho DANE & Categórico & - & - & - & 34 \\
  18 & Código Dane Departamento & Numérico & 5 & 999 & 994 & 34 \\
  19 & Escenario del Hecho & Categórico & - & - & - & 53 \\
  20 & Zona del Hecho & Categórico & - & - & - & 4 \\
  21 & Actividad Durante el Hecho & Categórico & - & - & - & 18 \\
  22 & Circunstancia del Hecho & Categórico & - & - & - & 50 \\
  23 & Manera de muerte & Categórico & - & - & - & 1 \\
  24 & Mecanismo Causal & Categórico & - & - & - & 18 \\
  25 & Diagnóstico Topográfico de la Lesión & Categórico & - & - & - & 11 \\
  26 & Presunto Agresor & Categórico & - & - & - & 53 \\
  27 & Condición de la Víctima & Categórico & - & - & - & 1 \\
  28 & Medio de Desplazamiento o Transporte & Categórico & - & - & - & 1 \\
  29 & Servicio del Vehículo & Categórico & - & - & - & 1 \\
  30 & Clase o Tipo de Accidente & Categórico & - & - & - & 1 \\
  31 & Objeto de Colisión & Categórico & - & - & - & 1 \\
  32 & Servicio del Objeto de Colisión & Categórico & - & - & - & 1 \\
  33 & Razón del Suicidio & Categórico & - & - & - & 1 \\
  34 & Localidad del Hecho & Categórico & - & - & - & 23 \\
  35 & Ancestro Racial & Categórico & - & - & - & 8 \\
\hline
\end{tabularx}
\end{table*}
  

\subsection{Dispersión y Resolución}

\begin{itemize}
    \item Alta dimensionalidad categórica, muchas columnas con texto y valores repetitivos.
    \item Pocas variables numéricas; dispersión estadística limitada.
    \item Buena resolución semántica en edad, ciclo vital, escolaridad.
    \item Precisión temporal anual; sin fechas exactas.
\end{itemize}
  	  

\subsection{Objetivos del Análisis}

\begin{itemize}
    \item Analizar evolución de homicidios a través del tiempo.
    \item Detectar patrones por edad, sexo, ubicación o ciclo vital.
    \item Predecir la circunstancia del hecho, presunto agresor o mecanismo causal.
    \item Predecir características del homicidio según otras variables.
\end{itemize}

\section{Análisis Exploratorio del Conjunto de Datos}
% Aquí se desarrollará el análisis exploratorio con gráficos, frecuencias, etc.

\section{Preprocesamiento del Conjunto de Datos}
% Aquí se describirá el manejo de valores nulos, codificación, escalamiento, etc.


\begin{thebibliography}{1}
\bibliographystyle{IEEEtran}

\bibitem{ref1}
{\it{Mathematics Into Type}}. American Mathematical Society. [Online]. Available: https://www.ams.org/arc/styleguide/mit-2.pdf

\bibitem{ref2}
T. W. Chaundy, P. R. Barrett and C. Batey, {\it{The Printing of Mathematics}}. London, U.K., Oxford Univ. Press, 1954.

\end{thebibliography}


\newpage



\vfill

\end{document}


