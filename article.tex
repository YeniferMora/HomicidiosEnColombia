\documentclass[lettersize,journal]{IEEEtran}
\usepackage{amsmath,amsfonts}
\usepackage{algorithmic}
\usepackage{algorithm}
\usepackage{array}
\usepackage[caption=false,font=normalsize,labelfont=sf,textfont=sf]{subfig}
\usepackage{textcomp}
\usepackage{stfloats}
\usepackage{url}
\usepackage{verbatim}
\usepackage{graphicx}
\usepackage{cite}
\usepackage{tabularx}

\hyphenation{op-tical net-works semi-conduc-tor IEEE-Xplore}
% updated with editorial comments 8/9/2021

\begin{document}

\title{A Sample Article Using IEEEtran.cls\\ for IEEE Journals and Transactions}

\author{IEEE Publication Technology,~\IEEEmembership{Staff,~IEEE,}
        % <-this % stops a space
\thanks{This paper was produced by the IEEE Publication Technology Group. They are in Piscataway, NJ.}% <-this % stops a space
\thanks{Manuscript received April 19, 2021; revised August 16, 2021.}}

% The paper headers
\markboth{Journal of \LaTeX\ Class Files,~Vol.~14, No.~8, August~2021}%
{Shell \MakeLowercase{\textit{et al.}}: A Sample Article Using IEEEtran.cls for IEEE Journals}

\IEEEpubid{0000--0000/00\$00.00~\copyright~2021 IEEE}
% Remember, if you use this you must call \IEEEpubidadjcol in the second
% column for its text to clear the IEEEpubid mark.

\maketitle

\begin{abstract}
This document describes the most common article elements and how to use the IEEEtran class with \LaTeX \ 
\end{abstract}

\begin{IEEEkeywords}
Article submission, IEEE, IEEEtran, journal, \LaTeX, paper, template, typesetting.
\end{IEEEkeywords}

\section{Introduction}
\IEEEPARstart{T}{his} file is intended to serve as a ``sample article file''
for IEEE journal papers produced under \LaTeX\ using
IEEEtran.cls version 1.8b and later. 

%---------------------------------------------------------------------------------
% Secciones del Documento
%---------------------------------------------------------------------------------

\section{Descripción del Conjunto de Datos}


\section{Análisis Exploratorio del Conjunto de Datos}

El objetivo de esta sección es realizar un Análisis Exploratorio de Datos (AED) sobre el conjunto de presuntos homicidios en Colombia (2015–2023), con énfasis en las variables categóricas (ya que estas conforman la mayoría de los datos). El notebook adjunto implementa una serie de pasos estándar y avanzados para:

\begin{itemize}
    \item Conocer la estructura y calidad del dataset.
    \item Identificar valores faltantes y la manera en la que estos están representados.
    \item Caracterizar variables numéricas y, especialmente, categóricas.
    \item Explorar relaciones bivariantes y multivariantes en las variables consideradas como más relevantes.
    \item Detectar categorías raras que puedan afectar futuros modelos.
\end{itemize}

Finalmente, a partir de todos estos hallazgos, se propone la variable \textbf{Circunstancia del Hecho} como la más adecuada para un futuro modelo de clasificación.

\subsection{Estrategias Utilizadas}

\subsubsection{Instalación y carga de librerías}
\begin{itemize}
    \item Se instaló la librería \texttt{tabulate} para presentar tablas de resumen en formato legible dentro de Colab.
    \item Se importaron \texttt{pandas}, \texttt{NumPy}, \texttt{matplotlib} y \texttt{seaborn}.
    \item Se habilitó \texttt{\%matplotlib inline} para ver los gráficos directamente en el notebook.
\end{itemize}

\subsubsection{Importación y vista inicial del dataset}
\begin{itemize}
    \item Lectura del CSV desde Google Drive, mediante \texttt{pd.read\_csv()}.
    \item Se utilizó \texttt{df.head()} para inspeccionar las primeras filas y \texttt{df.shape} para conocer su tamaño (111263 filas y 35 columnas).
    \item Con \texttt{df.info()} y \texttt{df.dtypes} se validaron los tipos de datos de cada columna (la mayoría como \texttt{object}, algunas numéricas como año, ID y Códigos DANE).
\end{itemize}

\subsubsection{Detección y tratamiento de valores faltantes}
\begin{itemize}
    \item Ejecución de \texttt{df.isnull().sum().sort\_values()} para obtener el conteo de verdaderos “NaN”.
    \item Según nuestras observaciones, pudimos identificar que muchos faltantes no se codifican como NaN sino como cadenas (“No aplica”, espacios en blanco, etiquetas locales). Teniendo en cuenta esto se realizó una inspección manual de los valores únicos para cada variable, buscando patrones de ausencias no estándar.
\end{itemize}

\subsubsection{Análisis univariante}

\paragraph{Variables numéricas}
\begin{itemize}
    \item Uso de \texttt{df.describe()} para resumir las variables numéricas, pero de esta solo se encontraron resultados relevantes en el año del hecho.
    \item Boxplots de edad judicial categorizada por sexo, revelando que la mediana y el rango intercuartílico de la edad difieren ligeramente entre hombres y mujeres.
\end{itemize}

\paragraph{Variables categóricas}
\begin{itemize}
    \item Identificación automática de variables de tipo objeto con \texttt{df.select\_dtypes(include='object')}.
    \item Frecuencias absolutas y relativas impresas con \texttt{value\_counts(dropna=False)} y presentadas con \texttt{tabulate} para mayor legibilidad.
    \item Gráficos de barras (\texttt{sns.countplot} sobre \texttt{matplotlib}) para las variables categóricas.
\end{itemize}

\subsubsection{Análisis bivariante y pruebas de asociación}
\begin{itemize}
    \item Tablas de contingencia (\texttt{pd.crosstab}) entre pares de categorías (Sexo vs. Circunstancia).
    \item Prueba de Chi-cuadrado (\texttt{chi2\_contingency}) en cada tabla para cuantificar si la relación observada es estadísticamente significativa.
\end{itemize}

\subsubsection{Detección de categorías raras}
\begin{itemize}
    \item Para cada variable categórica, se calcularon las proporciones de cada etiqueta y se listaron aquellas con frecuencia $<$ 1\% del total.
    \item Estas categorías de baja representación se consideraron para agruparse en un nivel “Otros” antes de cualquier modelado predictivo, con el fin de evitar ruido y problemas de \textit{sparsity}.
\end{itemize}

\subsubsection{Análisis multivariante y correlaciones}
\begin{itemize}
    \item Construcción de la matriz de correlación sobre las variables numéricas y visualización con \texttt{heatmap} de \texttt{seaborn} para identificar posibles colinealidades.
    \item Al observar que las variables numéricas no presentaban correlaciones fuertes, se confirmó que el análisis se debía centrar en las categorías.
\end{itemize}

\subsection{Principales Resultados}

\subsubsection{Calidad de los datos}
\begin{itemize}
    \item Aunque \texttt{df.info()} mostraba pocos NaN reales, se detectaron múltiples cadenas (“No aplica”, “Desconocido”, “no informado”) que requieren un preprocesamiento especial.
    \item Sólo la variable numérica (Año del hecho) resultó robusta; el resto eran categorías puras.
\end{itemize}

\subsubsection{Distribución de casos}
\begin{itemize}
    \item Se evidencia un aumento en los homicidios en el transcurso de los años. Se ve una tendencia incremental leve, con una excepción en el año 2020, donde se rompió esta tendencia incremental y se mostró valores muy similares a los del año 2017.
    \item Arma de fuego y arma corto punzante fueron mecanismos y circunstancias dominantes, con más del 60 \% de los casos cada uno.
\end{itemize}

\subsubsection{Categorías raras}
\begin{itemize}
    \item Se identificaron en la mayoría de variables etiquetas con frecuencias relativas por debajo del 1 \% (por ejemplo, algunas clases de “Objeto de colisión” en homicidios por transporte).
    \item En estas se debe considerar agrupaciones bajo categorías como “Otros” para mejorar la estabilidad de futuros modelos.
\end{itemize}

\subsection{Hallazgos}

El AED confirmó que la variable \textbf{Circunstancia del Hecho} es la más apropiada como objetivo para un modelo de clasificación, porque:

\begin{itemize}
    \item Tiene múltiples clases significativas (homicidio simple, agravado, riña, feminicidio, etc.).
    \item Muestra asociaciones estadísticamente robustas con otras variables clave (sexo, zona, escenario).
    \item Su predicción permitiría orientar políticas de prevención diferenciales según el contexto del homicidio.
\end{itemize}

Los pasos de detección de valores faltantes no estándar y de agrupación de categorías raras fueron críticos para asegurar la calidad de los datos. 

Se debe analizar qué tantas implicaciones e imposibilidades traerá el hecho que muchos atributos contienen valores faltantes hasta del 64\%.


\section{Preprocesamiento del Conjunto de Datos}


\begin{thebibliography}{1}
\bibliographystyle{IEEEtran}

\bibitem{ref1}
{\it{Mathematics Into Type}}. American Mathematical Society. [Online]. Available: https://www.ams.org/arc/styleguide/mit-2.pdf

\bibitem{ref2}
T. W. Chaundy, P. R. Barrett and C. Batey, {\it{The Printing of Mathematics}}. London, U.K., Oxford Univ. Press, 1954.

\end{thebibliography}


\newpage



\vfill

\end{document}

